\section{Introduction}
Binary response models have a very rich history in the statistical research literature with diverse
applications in many fields where the dependent variable takes on a dichotomous value. Early work on these
models included the tetrachoric correlation analysis of \cite{Pearson1900} and the standard biometric textbook
of \cite{Finney1971}, with more recent advances dealing with such issues as full or partial separation
in \cite{Zorn2005}, small sample bias reduction in \cite{Firth1993} and \cite{Heinze2002} and
heteroscedasticity (see for instance \cite{Keele2006}). In econometrics, \cite{Amemiya1981} provided an early
review of applications, but more recent research has focused on direction of change forecast for stock market
returns and recession forecasting which is also the motivation of this author.\\
The Dynamic Binary Model (\verb@dbm@) package implements an autoregressive binary regression model described
in \cite{Kauppi2008} and further discussed and extended in \citeauthor*{Nyberg2011}(\citeyear{Nyberg2010}, \citeyear{Nyberg2010a},
and \citeyear{Nyberg2011}). \proglang{R} already includes a large number of packages implementing dynamic binary models,
with basic functionality already available in the \emph{glm} function of the \verb@stats@ package, and more advanced
variations in \cite{Zeileis2011}(skew-or generalized-logit), \cite{Zeileis2011a} (heteroscedasticy),
and \cite{Heinze2013} (bias reduction methods), to name but a few. The \verb@dbm@ package uniquely implements the
autoregressive model of \cite{Kauppi2008} which is believed to provide a much improved fit to typical problems in
the time series domain.\\
The vignette discusses the model background in Section \ref{sec:background} and its implementation in
Section \ref{sec:implementation}. An interesting application, and exposition of the package's functionality
is available on my blog (see below). The appendix derives the analytic gradient expressions used in
the numeric optimization routines.\\

The \verb@dbm@ package is currently available in the \emph{teatime} package repository
on r-forge\\
(\url{http://cran.r-project.org/web/packages/teatime/index.html}).
An example is available on my blog (\url{http://www.unstarched.net}).\\

The package is provided AS IS, without any implied warranty as to its accuracy
or suitability. A lot of time and effort has gone into the development of this
package, and it is offered under the GPL-3 license in the spirit of open
knowledge sharing and dissemination. If you do use the model in published work
DO remember to cite the package and author (type \verb@citation@("dbm") for
the appropriate BibTeX entry) , and if you have used it and found it
useful, drop me a note and let me know.\\

\textbf{USE THE R-SIG-FINANCE MAILING LIST FOR QUESTIONS.}
\section{Background}\label{sec:background}
Consider the stochastic process $y_t$, which is binary valued, and the vector of $k$ explanatory variables,
$x_{i,t}$ for $i=1\ldots k$. Let $\Im_t$ be the information set available at time $t$, then $y_t$
has a Bernoulli distribution with probability $p_t$:
\begin{equation}
{y_t}\left| {{\Im _{t - 1}} \sim B\left( {{p_t}} \right)} \right.
\end{equation}
The objective is to model $p_t$ through the CDF transformed dynamics\footnote{The CDF transformation is monotonically increasing and
guarantees that the resulting transformation will be in the unit interval.}  of a linear process $\pi_t$ such that ${p_t} = \Phi \left( {{\pi _t}} \right)$.
Formally,
\begin{equation}
{E_{t - 1}}\left( {{y_t}} \right) = {P_{t - 1}}\left( {y = 1} \right) = \Phi {\text{ }}\left( {{\pi _t}} \right) = {p_t}
\end{equation}
The CDF function, also called the link function, can be one of any number of distributions but has typically been either the Gaussian (probit model),
Logistic (logit model) or some skewed variation (scobit model). In the \verb@dbm@ package, the Generalized Logistic distribution has been chosen as a skewed choice since
it nests the Logistic. In the model of \cite{Kauppi2008}, the dynamics of $\pi_t$ take on the following form:
\begin{equation}\label{eq:dbm}
{\pi _t} = \omega  + \sum\limits_{i = 1}^k {{\beta _i}{x_{i,t - {l_i}}}}  + \sum\limits_{i = 1}^q {{\delta _i}{y_{t - i}}}  + \sum\limits_{i = 1}^p {{\alpha _i}{\pi _{t - i}}}
\end{equation}
where $\delta_i$ represents the coefficient on the $q$ autoregressive terms of the binary variable $y_t$, $\beta_i$ the coefficient on the $i^{th}$(of $k$) explanatory variable $x_t$ with lag $l_i$\footnote{The representation used here, which is the one used in the package, is such that the vector $\mathbf{x}_t$ can contain the same explanatory variable but with a different lag thus enabling a great degree of flexibility, albeit at the cost of some redundancy, in the dynamics.}, and $\alpha_i$ the coefficient on the $p$ autoregressive terms of the dynamics $\pi_t$. The specification without the latter term has already been examined elsewhere, and some results with regards to its asymptotic properties can be found in \cite{Jong2011}.  Related literature on general binomial ARMA type models can be found, among others, in \cite{Al-Osh1991} and more recently, with financial/econometric applications, in \cite{Rydberg2003} and \cite{Startz2008}. \cite{Nyberg2011} introduced a restriction to Equation \ref{eq:dbm} by setting $\delta_1=1-\alpha_1$, leading to a type of error correction model (\emph{ecm}) with strong persistence in the autoregressive dynamics parameter $\alpha_1$ usually observed.\\
\subsection{Maximum Likelihood Estimation}
The log-likelihood function (at time $t$) in the maximization, given the vector of parameters $\theta$, the conditional dynamics ${\pi _t}\left( \theta  \right)$ and an appropriate link function $\Phi$, can be represented as:
\begin{equation}
l\left( \theta  \right) = \sum\limits_{t = 1}^T {\left[ {{y_t}\log \Phi \left( {{\pi _t}\left( \theta  \right)} \right) + \left( {1 - {y_t}} \right)\log \left( {1 - \Phi \left( {{\pi _t}\left( \theta  \right)} \right)} \right)} \right]}
\end{equation}
conditional on initial values for the recursion. Appendix \ref{sec:Appendix} contains the details of the gradient for each of the 3 link functions used in the \verb@dbm@ package and the recursion initialization method.\\
\subsection{Forecasting}
Unlike other nonlinear models, the binary nature of the model allows explicit multi-period iterated forecasts by enumeration of all the possible binary paths. Following from the Appendix of \cite{Kauppi2008}, and adjusting the notation to use forward times, the h-period ahead forecast can be represented as follows:
\begin{equation}
\begin{aligned}
  {E_t}\left( {{y_{t + h}}} \right) &= {E_t}\Phi \left( {{\alpha ^h}{\pi _t} + \sum\limits_{j = 1}^h {\left[ {{\alpha ^{j - 1}}\left( {\omega  + \delta {y_{t + h - j}} + \sum\limits_{i = 1,(h - {l_i}) < j}^p {{\beta _i}{x_{i,t + \left( {h - {l_i}} \right) + 1 - j}}} } \right)} \right]} } \right) \hfill \\
   &= \sum\limits_{y_{t + 1}^{t + h - 1} \in {B_{h - 1}}} {{P_t}\left( {y_{t + 1}^{t + h - 1}} \right)} \Phi \left( {{\alpha ^h}{\pi _t} + \sum\limits_{j = 1}^h {\left[ {{\alpha ^{j - 1}}\left( {\omega  + \delta {y_{t + h - j}} + \sum\limits_{i = 1,L\left( {{x_i}} \right) \geqslant j}^l {{\beta _i}{x_{i,t + 1 - j}}} } \right)} \right]} } \right) \hfill \\
\end{aligned}
\end{equation}
where, ${y_{t + 1}^{t + h - 1} \in {B_{h - 1}}}$ indicates the evaluation of all possible binary paths for $y$ up to time $t+h-1$, $l_i$ is the lag of each explanatory variable $x_i,i=1,\ldots k$, and
\begin{equation}
\begin{gathered}
  {P_t}\left( {y_{t + 1}^{t + h - 1}} \right) = \prod\limits_{n = 1}^{h - 1} {{{\left( {{p_{t + n}}} \right)}^{{y_{t + n}}}}{{\left( {1 - {p_{t + n}}} \right)}^{\left( {1 - {y_{t + n}}} \right)}}}  \hfill \\
  {p_{t + n}} = \Phi \left( {{\alpha ^n}{\pi _t} + \sum\limits_{j = 1}^n {{\alpha ^{j - 1}}\left( {\omega  + \delta {y_{t - 1 + j}} + \sum\limits_{i = 1,{l_i} \geqslant j}^k {{\beta _i}{x_{\left\{ {i,t - {l_i} + j} \right\}}}} } \right)} } \right) \hfill \\
\end{gathered}
\end{equation}
Some caution is warranted here since the multi-path forecast, while explicit, starts to grow very fast. Because the implementation in the \verb@dbm@ package does not enforce any compact representation (it simply makes use of the \emph{expand.grid} function), memory issues are likely to arise for forecast horizons greater than 15.
\section{Implementation}\label{sec:implementation}
The dbm package estimates the model of \cite{Kauppi2008} by maximum likelihood using analytic gradient information which is passed to the appropriate solver.
The default is to use the Nelder-Mead algorithm from the optim package with the use of a bounding logistic transformation for the parameters which are constrained
to ensure either stationarity (autoregressive parameter $\alpha$) or existence in the real domain (the Generalized Logistic skew parameter
$skew[k]$).
The result of using such a transformation (given that the optim package does not necessarily include lower and upper bounds for all but one solver), is that the gradient of
those parameters are shifted slightly from the unconstrained case and I make no special provision to account for this. Since the transformation is turned off
during the calculation of the final hessian and scores, the standard errors and resulting information is not affected, and it is also my experience that
estimation is not in the slightest affected by this. If in doubt, there is always the possibility of using any of bound constrained solvers from the nloptr package
with analytic gradient or the gosolnp solver from the Rsolnp package with numerical gradient. Also, for the \textbf{ecm} case, the analytic derivatives are turned
off since I have not yet implemented special consideration for this. Finally, both the likelihood and gradient, as well as the main part of the forecast routine is
implemented in C and C++ for speed.\\
The main functionality can be found in the \verb@dbm@ function which is used for the ML estimation of a model:
\begin{Schunk}
\begin{Sinput}
> args(dbm)
\end{Sinput}
\begin{Soutput}
function(y, x.vars = NULL, x.lags = 1, arp = 1, arq = 0, ecm = FALSE,
	constant = TRUE, link = "gaussian", fixed.pars = NULL,
	solver = "optim", control=list(), parsearch = TRUE, parsim = 5000,
	method = "Nelder-Mead", ...)
\end{Soutput}
\end{Schunk}
The first argument, \emph{y}, is a multivariate xts matrix with the first column being the binary dependent variable, and the \textbf{names} of the
independent explanatory variables (corresponding to the column names in y) are passed via the \emph{x.vars} character vector. The lags for each
of the explanatory variables is then passed using the x.lags integer vector. This means that if you want to include multiple lags from the same
variable, then the y matrix must include that same variable data as many times as the lags required (with different names). Also note that you must
not pass any of the data lagged since that is done by the routine (which is why \emph{x.lags} is used). The \emph{arp} and \emph{arp} options denote
the lags for the auto-regression in the dynamics and dependent variable, respectively, but as currently implemented only lag 1 is allowed. The \emph{ecm}
option indicates whether to use the error correction restriction of \cite{Nyberg2011}, \emph{constant} whether to include an intercept, and \emph{link}
the CDF link function with a choice of 'gaussian' (probit), 'logistic' (logit) and 'glogistic' (scobit). The other options are related to
the parameter start, fixed values (as a named vector), the choice of solver etc. Standard extraction and inference methods such as \emph{fitted}, \emph{residuals},
\emph{coef}, \emph{vcov} (with choice of robust), \emph{likelihood}, \emph{deviance} (with null model choice), \emph{plot} and \emph{summary}. Additionally,
a score method is also included to extract either the analytic or numeric scores with the option of also recalculating them for a new set of parameters.
This is particularly useful when performing a Lagrange Multiplier (\emph{LM}) test which requires the score of the fixed parameter. An example of this and
additional functions are documented in the package help.\\
Finally, it is up to the user to determine the adequacy of the estimated model. There are a lot of packages on CRAN which implement diagnostic tests, and
the \verb@dbm@ package returns enough information in the estimated object (including the scores) in order to run most of the available tests. There are a
number of important issues to keep in mind when estimating a model, including heteroscedasticity, partial or complete non-separation, and
multi-collinearity, all of which are covered in standard textbooks and numerous articles. Particular care should also be taken to ensure that
the explanatory variables are stationary.
\section{Appendix: Analytic Derivatives}\label{sec:Appendix}
The appendix provides a dry exposition of the analytic expressions for the derivatives used in the numerical
optimization routine for the 3 link functions. The subscripts for the parameter lag orders and explanatory
variables have been suppressed for ease and compactness of notation, and denote a lag-1 in all variables with
one explanatory variable, but the results easily generalize to higher lags and k explanatory variables.
\subsection{Recursion Initialization}
The recursion is initialized by setting:
\begin{equation}\label{eq:rcs}
{\pi _0} = \frac{{\omega  + \delta \bar y + \beta \bar x}}{{1 - \alpha }}
\end{equation}
where the bar over the variables denotes their unconditional mean. Therefore, at time $t_1$, ${\pi _1} = \omega  + \alpha {\pi _0}$, and the partial derivatives with respect to $\pi_1$ are:
\begin{equation}\label{eq:rcsderiv}
\begin{gathered}
  \frac{{\partial {\pi _1}}}
{{\partial \omega }} = \frac{\alpha }
{{1 - \alpha }} + 1 \hfill \\
  \frac{{\partial {\pi _1}}}
{{\partial \beta }} = \frac{{\alpha \bar x}}
{{1 - \alpha }} \hfill \\
  \frac{{\partial {\pi _1}}}
{{\partial \delta }} = \frac{{\alpha \bar y}}
{{1 - \alpha }} \hfill \\
  \frac{{\partial {\pi _1}}}
{{\partial \alpha }} = {\pi _0} + \frac{{\alpha {\pi _0}}}
{{1 - \alpha }} \hfill \\
\end{gathered}
\end{equation}
\subsection{Logistic}
The logistic link function, which gives rise to the \emph{logit} model, has the following log-likelihood function at time $t$ (maximization):
\begin{equation}
{l_t}\left( \theta  \right) = {y_t}\log \left( {\frac{1}
{{1 + {e^{ - {\pi _t}}}}}} \right) + \left( {1 - {y_t}} \right)\log \left( {1 - \frac{1}
{{1 + {e^{ - {\pi _t}}}}}} \right)
\end{equation}
The partial derivatives with respect to the log-likelihood are defined as follows:
\begin{equation}
\begin{gathered}
  \frac{{\partial {l_t}}}
{{\partial \omega }} = \frac{{\left( {{y_t} + {y_t}{{\text{e}}^{ - {\pi _t}}} - 1} \right)}}
{{\left( {1 + {{\text{e}}^{ - {\pi _t}}}} \right)}}\left( {1 + \frac{{\partial {\pi _{t - 1}}\left( \theta  \right)}}
{{\partial \omega }}} \right) \hfill \\
  \frac{{\partial {l_t}}}
{{\partial \beta }} = \frac{{\left( {{y_t} + {y_t}{{\text{e}}^{ - {\pi _t}}} - 1} \right)}}
{{\left( {1 + {{\text{e}}^{ - {\pi _t}}}} \right)}}\left( {{x_{t - 1}} + \alpha \frac{{\partial {\pi _{t - 1}}\left( \theta  \right)}}
{{\partial {x_{t - 1}}}}} \right) \hfill \\
  \frac{{\partial {l_t}}}
{{\partial \delta }} = \frac{{\left( {{y_t} + {y_t}{{\text{e}}^{ - {\pi _t}}} - 1} \right)}}
{{\left( {1 + {{\text{e}}^{ - {\pi _t}}}} \right)}}\left( {{y_{t - 1}} + \alpha \frac{{\partial {\pi _{t - 1}}\left( \theta  \right)}}
{{\partial {\delta _{t - 1}}}}} \right) \hfill \\
  \frac{{\partial {l_t}}}
{{\partial \alpha }} = \frac{{\left( {{y_t} + {y_t}{{\text{e}}^{ - {\pi _t}}} - 1} \right)}}
{{\left( {1 + {{\text{e}}^{ - {\pi _t}}}} \right)}}\left( {{\pi _{t - 1}} + \alpha \frac{{\partial {\pi _{t - 1}}\left( \theta  \right)}}
{{\partial \alpha }}} \right) \hfill \\
\end{gathered}
\end{equation}
The ${\partial {\pi _{t - 1}}\left( \theta  \right)}$ is based on the initialization values and their partial derivatives given in Equation \ref{eq:rcsderiv}.
\subsection{Gaussian}
The Gaussian link function, which gives rise to the \emph{probit} model, has the following log-likelihood function at time $t$ (maximization):
\begin{equation}
{l_t}\left( \theta  \right) = {y_t}\ln \left( {\frac{1}
{2} + \frac{1}
{2}\,{\text{erf}}\left( {\frac{{{\pi _t}}}
{{\sqrt 2 }}} \right)} \right) + \left( {1 - {y_y}} \right)\ln \left( {\frac{1}
{2} - \frac{1}
{2}\,{\text{erf}}\left( {\frac{{{\pi _t}}}
{{\sqrt 2 }}} \right)} \right)
\end{equation}
where the error function $erf$ is used to approximate the Gaussian distribution. The recursion is initialized as in Equation \ref{eq:rcs}. The partial derivatives with respect to the log-likelihood are defined as follows:
\begin{equation}
\begin{gathered}
  \frac{{\partial {l_t}}}
{{\partial \omega }} = \frac{{{{\text{e}}^{ - 0.5\pi _t^2}}\sqrt 2 \left( {{\text{erf}}\left( { - 0.5\sqrt 2 \,{\pi _t}} \right) - 2\,{y_t}{\text{ + }}1} \right)}}
{{\sqrt {\mathbf{\pi }} \left( {{\text{erf}}{{\left( { - 0.5\sqrt 2 {\pi _t}} \right)}^2} - 1} \right)}}\left( {1 + \alpha \frac{{\partial {\pi _{t - 1}}\left( \theta  \right)}}
{{\partial \omega }}} \right) \hfill \\
  \frac{{\partial {l_t}}}
{{\partial \beta }} = \frac{{{{\text{e}}^{ - 0.5\pi _t^2}}\sqrt 2 \left( {{\text{erf}}\left( { - 0.5\sqrt 2 \,{\pi _t}} \right) - 2\,{y_t}{\text{ + }}1} \right)}}
{{\sqrt {\mathbf{\pi }} \left( {{\text{erf}}{{\left( { - 0.5\sqrt 2 {\pi _t}} \right)}^2} - 1} \right)}}\left( {{x_{t - 1}} + \alpha \frac{{\partial {\pi _{t - 1}}\left( \theta  \right)}}
{{\partial \beta }}} \right) \hfill \\
  \frac{{\partial {l_t}}}
{{\partial \delta }} = \frac{{{{\text{e}}^{ - 0.5\pi _t^2}}\sqrt 2 \left( {{\text{erf}}\left( { - 0.5\sqrt 2 \,{\pi _t}} \right) - 2\,{y_t}{\text{ + }}1} \right)}}
{{\sqrt {\mathbf{\pi }} \left( {{\text{erf}}{{\left( { - 0.5\sqrt 2 {\pi _t}} \right)}^2} - 1} \right)}}\left( {{y_{t - 1}} + \alpha \frac{{\partial {\pi _{t - 1}}\left( \theta  \right)}}
{{\partial \delta }}} \right) \hfill \\
  \frac{{\partial {l_t}}}
{{\partial \alpha }} = \frac{{{{\text{e}}^{ - 0.5\pi _t^2}}\sqrt 2 \left( {{\text{erf}}\left( { - 0.5\sqrt 2 \,{\pi _t}} \right) - 2\,{y_t}{\text{ + }}1} \right)}}
{{\sqrt {\mathbf{\pi }} \left( {{\text{erf}}{{\left( { - 0.5\sqrt 2 {\pi _t}} \right)}^2} - 1} \right)}}\left( {{\pi _{t - 1}}\left( \theta  \right) + \alpha \frac{{\partial {\pi _{t - 1}}\left( \theta  \right)}}
{{\partial \alpha }}} \right) \hfill \\
\end{gathered}
\end{equation}
The ${\partial {\pi _{t - 1}}\left( \theta  \right)}$ is based on the initialization values and their partial derivatives given in Equation \ref{eq:rcsderiv}.
\subsection{Generalized Logistic}
The Generalized Logistic distribution allows for the asymmetric impact of the explanatory variables with the following distribution
function:
\begin{equation}
\Phi \left( {{y_t}} \right) = {\left( {1 + {e^{ - {\pi_t}}}} \right)^{ - k}},
\end{equation}
with $k \in {\Re ^ + }$ and $\pi_t$ representing the dynamics given in Equation \ref{eq:dbm}. This has also been called the \emph{scobit}
model\footnote{See for instance \cite{Zeileis2011} for \emph{one} implementation.} since it nests the logistic distribution when
$k=1$.
The log-likelihood to be maximized in the binary response model is then given by the following equation at time $t$:
\begin{equation}
{l_t}\left( \theta  \right) = {y_t}\log \left( {{{\left( {1 + {e^{ - {\pi _t}}}} \right)}^{ - k}}} \right) + \left( {1 - {y_t}} \right)\log \left( {1 - {{\left( {1 + {e^{ - {\pi _t}}}} \right)}^{ - k}}} \right)
\end{equation}
The partial derivatives with respect to the log-likelihood are defined as follows:
\begin{equation}
\begin{gathered}
  \frac{{\partial {l_t}\left( \theta  \right)}}
{{\partial \omega }} = \frac{{k{{\text{e}}^{ - {\pi _t}}}\left( {{{\text{e}}^{k{\pi _t}}} - {y_t}{{\left( {{\text{1 + }}{{\text{e}}^{{\pi _t}}}} \right)}^k}} \right)}}
{{\left( {1 + {{\text{e}}^{ - {\pi _t}}}} \right)\left( {{{\text{e}}^{k{\pi _t}}} - {{\left( {{\text{1 + }}{{\text{e}}^{{\pi _t}}}} \right)}^k}} \right)}}\left( {1 + \alpha \frac{{\partial {\pi _{t - 1}}\left( \theta  \right)}}
{{\partial \omega }}} \right) \hfill \\
  \frac{{\partial {l_t}\left( \theta  \right)}}
{{\partial \beta }} = \frac{{k{{\text{e}}^{ - {\pi _t}}}\left( {{{\text{e}}^{k{\pi _t}}} - {y_t}{{\left( {{\text{1 + }}{{\text{e}}^{{\pi _t}}}} \right)}^k}} \right)}}
{{\left( {1 + {{\text{e}}^{ - {\pi _t}}}} \right)\left( {{{\text{e}}^{k{\pi _t}}} - {{\left( {{\text{1 + }}{{\text{e}}^{{\pi _t}}}} \right)}^k}} \right)}}\left( {{x_{t - 1}} + \alpha \frac{{\partial {\pi _{t - 1}}\left( \theta  \right)}}
{{\partial \beta }}} \right) \hfill \\
  \frac{{\partial {l_t}\left( \theta  \right)}}
{{\partial \delta }} = \frac{{k{{\text{e}}^{ - {\pi _t}}}\left( {{{\text{e}}^{k{\pi _t}}} - {y_t}{{\left( {{\text{1 + }}{{\text{e}}^{{\pi _t}}}} \right)}^k}} \right)}}
{{\left( {1 + {{\text{e}}^{ - {\pi _t}}}} \right)\left( {{{\text{e}}^{\text{k}}}{\pi _{\text{t}}}{\text{ - }}{{\left( {{\text{1 + }}{{\text{e}}^{{\pi _{\text{t}}}}}} \right)}^{\text{k}}}} \right)}}\left( {{y_{t - 1}} + \alpha \frac{{\partial {\pi _{t - 1}}\left( \theta  \right)}}
{{\partial \delta }}} \right) \hfill \\
  \frac{{\partial {l_t}\left( \theta  \right)}}
{{\partial \alpha }} = \frac{{k{{\text{e}}^{ - {\pi _t}}}\left( {{{\text{e}}^{k{\pi _t}}} - {y_t}{{\left( {{\text{1 + }}{{\text{e}}^{{\pi _t}}}} \right)}^k}} \right)}}
{{\left( {1 + {{\text{e}}^{ - {\pi _t}}}} \right)\left( {{{\text{e}}^{k{\pi _t}}} - {{\left( {{\text{1 + }}{{\text{e}}^{{\pi _t}}}} \right)}^k}} \right)}}\left( {{\pi _{t - 1}}\left( \theta  \right) + \alpha \frac{{\partial {\pi _{t - 1}}\left( \theta  \right)}}
{{\partial \alpha }}} \right) \hfill \\
  \frac{{\partial {l_t}\left( \theta  \right)}}
{{\partial k}} =  - \frac{{\log \left( {1 + {{\text{e}}^{ - {\pi _t}}}} \right)\left( {{{\text{e}}^{k{\pi _t}}} - {y_t}{{\left( {{\text{1 + }}{{\text{e}}^{{\pi _t}}}} \right)}^k}} \right)}}
{{{{\text{e}}^{k{\pi _t}}} - {{\left( {{\text{1 + }}{{\text{e}}^{{\pi _t}}}} \right)}^k}}} \hfill \\
\end{gathered}
\end{equation}
The ${\partial {\pi _{t - 1}}\left( \theta  \right)}$ is based on the initialization values and their partial derivatives given in Equation \ref{eq:rcsderiv}.
